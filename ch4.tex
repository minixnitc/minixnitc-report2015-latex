\chapter{Further Work}

\section {About Website}
We have created a website - \url{minixnitc.github.io} so that all the further projects done on minix can be compiled here. All that we learned while working on the project is posted here.

\subsection{Guide to Minix}
This section consists of notes from the textbook -Design and Implementation of Operating Systems by Andrew Tanenbaum. Sicne this book is based on earlier versions of minix, we have included our own understanding of the code wherever we found that the system deviates from the test. Because on inclusion of Virtial File System in MINIX 3.2, there is a considerable difference in File System and System Call implementation.
\subsection{Hands-on Tutorials on Minix OS}
\begin{itemize}
\item \url{www.cs.ucsb.edu/~ravenben/classes/170/html/projects.html}
\item \url{web.fe.up.pt/~pfs/aulas/lcom2014/labs/doc/MinixMacVB.pdf}
\item \url{www.cis.syr.edu/~wedu/seed/Documentation/Minix3/How_to_add_system_call.pdf}
\item \url{cise.ufl.edu/class/cop4600sp14/Minix-Syscall_Tutorialv2.pdf}
\item \url{www.phien.org/ucdavis/ta/ecs150-f03/syscall.html}
\item \url{wiki.minix3.org/doku.php?id=developersguide:newkernelcall}

\end{itemize}

\subsection {Implementation}
This section consists of detailed explaination of the functions in file system and VFS which were changed while implementing Immediate files.
The relevant code in implementation has also been explained in detail.

\section {Possible Projects}
Minix is an ever growing system and there are many projects that can be done based on it at btech level, since minix was basically developed for education purposes. Here are the links to Google Summer Of Code page for Minix

\begin{itemize}
\item GSOC 2011 - \url{http://wiki.minix3.org/doku.php?id=soc:2011:start}
\item GSOC 2012 - \url{http://wiki.minix3.org/doku.php?id=soc:2012:start}
\item GSOC 2013 - \url{http://wiki.minix3.org/doku.php?id=soc:2012:start}
\item One can further explore the system for more ideas
\end{itemize}

